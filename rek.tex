\documentclass[12pt]{article}
\usepackage{polski}
\usepackage{amsmath}
\usepackage{amsfonts}
\begin{document}
Wzór rekurencyjny ciągu - wzór pozwalający na obliczenie wyrazu ciągu za pomocą poprzedniego wyrazu. Bardzo popularnym przykładem jest ciąg Fibonacciego:
$$a_1=a_2=1,$$ 
$$a_n=a_{n-1}+a_{n-2}$$ 
Innymi słowy - dwa pierwsze wyrazy ciągu są równe $1$ a kolejne są sumą dwóch poprzednich tj. $$a_3=a_2+a_1=2,$$ $$a_4=a_3+a_2=3,$$ 
i tak dalej.\\
Rekurencyjnie można zdefiniować np. silnię liczby $n\in\mathbb{N}$ (ozn. $n!$) jako:
$$n!=\begin{cases}
1 & \text{gdy $n=1$} \\
n\cdot(n-1)! & \text{w przeciwnym wypadku}
\end{cases}$$
Jest to równoważne ogólnemu wzorowi:
$$n!=1\cdot 2\cdot\dots\cdot (n-1)\cdot n$$
Ogólna intuicja rekurencji jest taka, że mając dwa kolejne wyrazy ciągu trzeba napisać co należy zrobić z jednym żeby uzyskać następny.\\
W zadaniu mamy ciąg $a_n=-n^2-3n$. Jego kolejne wyrazy to $-4,-10,-18,-20,\dots$.
Mamy:
$$a_1=-4$$
$$a_2=-4-6=a_1-6$$
$$a_3=-10-8=a_2-8$$
Ogólnie w każdym kroku odejmujemy o $2$ więcej nic w poprzednim. Stąd uzyskujemy:
$$a_n=a_{n-1}-2(n+1)$$
$$CF_C=(\langle 1, 8\rangle,\langle 2, 8\rangle,\langle 3,8\rangle,\langle 4, 108\rangle)$$
\end{document}